\documentclass[]{article}
\usepackage{lmodern}
\usepackage{amssymb,amsmath}
\usepackage{ifxetex,ifluatex}
\usepackage{fixltx2e} % provides \textsubscript
\ifnum 0\ifxetex 1\fi\ifluatex 1\fi=0 % if pdftex
  \usepackage[T1]{fontenc}
  \usepackage[utf8]{inputenc}
\else % if luatex or xelatex
  \ifxetex
    \usepackage{mathspec}
    \usepackage{xltxtra,xunicode}
  \else
    \usepackage{fontspec}
  \fi
  \defaultfontfeatures{Mapping=tex-text,Scale=MatchLowercase}
  \newcommand{\euro}{€}
\fi
% use upquote if available, for straight quotes in verbatim environments
\IfFileExists{upquote.sty}{\usepackage{upquote}}{}
% use microtype if available
\IfFileExists{microtype.sty}{%
\usepackage{microtype}
\UseMicrotypeSet[protrusion]{basicmath} % disable protrusion for tt fonts
}{}
\usepackage[margin=1in]{geometry}
\ifxetex
  \usepackage[setpagesize=false, % page size defined by xetex
              unicode=false, % unicode breaks when used with xetex
              xetex]{hyperref}
\else
  \usepackage[unicode=true]{hyperref}
\fi
\hypersetup{breaklinks=true,
            bookmarks=true,
            pdfauthor={Kushal K Dey},
            pdftitle={ancient DNA: damage, contamination, PCR},
            colorlinks=true,
            citecolor=blue,
            urlcolor=blue,
            linkcolor=magenta,
            pdfborder={0 0 0}}
\urlstyle{same}  % don't use monospace font for urls
\setlength{\parindent}{0pt}
\setlength{\parskip}{6pt plus 2pt minus 1pt}
\setlength{\emergencystretch}{3em}  % prevent overfull lines
\setcounter{secnumdepth}{0}

%%% Use protect on footnotes to avoid problems with footnotes in titles
\let\rmarkdownfootnote\footnote%
\def\footnote{\protect\rmarkdownfootnote}

%%% Change title format to be more compact
\usepackage{titling}

% Create subtitle command for use in maketitle
\newcommand{\subtitle}[1]{
  \posttitle{
    \begin{center}\large#1\end{center}
    }
}

\setlength{\droptitle}{-2em}
  \title{ancient DNA: damage, contamination, PCR}
  \pretitle{\vspace{\droptitle}\centering\huge}
  \posttitle{\par}
  \author{Kushal K Dey}
  \preauthor{\centering\large\emph}
  \postauthor{\par}
  \predate{\centering\large\emph}
  \postdate{\par}
  \date{January 27, 2016}



\begin{document}

\maketitle


\subsection{Overview}\label{overview}

In the first chapter of this Journal Club, we shall consider three very
instrinsic features of ancient DNA (aDNA) analysis.

\begin{itemize}
\item DNA damage
\item DNA contamination 
\item PCR amplification and sequencing of aDNA 
\end{itemize}

We dicussed this
\href{http://journals.plos.org/plosone/article?id=10.1371/journal.pone.0034131}{paper}

\subsection{DNA damage}\label{dna-damage}

DNA gets damaged when an organism is alive but there are repair
mechanisms that can fix those damages. However, these repair mechanisms
do not function after the organism dies. As a result, few intact copies
of aDNA tend to survive in old specimens, and those that remain are
often highly fragmented and damaged. Preservation in cold environments
sometimes prevents damages. Hydrolytic damage leads to single-strand
breaks through direct cleavage or following depurination, fragmenting
the DNA. Hydrolysis can also cause \(C \rightarrow T\). DNA damage can
be braodly catregorized into following types.

\begin{itemize}
\item Strand breaks: occurs due to heat, chemicals, water and microbe degradation. Results in low quantity of surviving DNA with short fragment lengths. 
\item Miscoding lesions: mainly due to hydrolysis, $C \rightarrow T$, $G \rightarrow A (xanthine)$, $C \rightarrow U$ etc. 
\item DNA-DNA crosslinks or DNa-protein crosslinks that prevent amplification.
\end{itemize}

The \(G \rightarrow X\) transitions are more prominent at the \(3^{'}\)
end of the strand and \(C \rightarrow T\) transitions are more prominent
at the \(5^{'}\) end of the strand. There are enzymes that can detect
which of the \(U/T\)s have come from \(C\) and which from \(T\), and it
can splice up the DNA at these points. Also the DNA fragments we get are
shorter in length. There is an increase occurrence in A and G
nucleotides near the strand breaks. These features in DNA accumulate
over time and may be used to infer the age of the ancient samples. UDG
(uracil DNA glycosylase) to remove uracil and splice the ancient DNA. To
solve the problem of short fragments, we amplify the overlapping
fragments.

We usually have DNA fragments with fragment length \(<100 bp\) and
contain damaged bases. Cold, dry, temperature-stable environments such
as permafrost regions and caves are among the best sources of
well-preserved specimens and have permitted large-scale population
studies. The theoretical limit to DNA preservation remains between
100,000 and 1,000,000 years.

\subsection{DNA contamination}\label{dna-contamination}

The most serious obstacle to ancient DNA analysis is DNA contamination.
The high sensitivity of PCR allows amplification to proceed from only
one or a few starting copies of the target sequence, but also often
allows contaminating DNA to be amplified. Also, even if the
contamination is low, PCR will preferentailly amplify modern DNA against
ancient DNA. Copies of the targeted fragment may contain blocking
lesions or simply be in low abundance, so that it enters the exponential
phase of the PCR many cycles after the reaction has begun. Contamination
may occur in different stages.

\begin{itemize}
\item  Bones and teeth are porous, and contamination may occur via adherence or uptake of exogenous DNA, often from microorganisms residing in the depositional environment. 
\item Contamination may also occur from modern humans during collection.
\item Contamination may occur in lab through laboratory personnel or other reagents. So, contamination may occur at the extraction and amplification stages of the DNA.
\item  Previously amplified DNA present in the laboratory environment is a particularly dangerous source of contaminating DNA. Even the tiny amount of DNA that is aerosolized when a tube is opened is likely to contain over a million copies of template, that can suppress the ancient DNA completely.
\end{itemize}

We must also make sure when we sample the ancient DNA from some
specimens, that we do not harm the remains or the specimen or destroy
parts of it. Conservation is a key issue here. So the type of aDNA
sampling used is not destructive.

\subsection{Amplification and Cloning}\label{amplification-and-cloning}

The invention of the polymerase chain reaction (PCR) revolutionized the
field of ancient DNA (aDNA) research. In theory, only a single copy of
the targeted DNA region is required for PCR, making it a powerful tool
for amplifying aDNA from samples where only a handful of intact copies
of the target region may remain. PCR is used when we want to make
multiple copies of the DNA.

PCR takes as input a DNA. It denatures/separates the two strands by
applying heat (\(95^{\circ} C\) for 15 seconds). Heat usually destroys
the enzymes that would replicate these strands once they are separated.
The solution is cooled to \(72^{\circ} C\) and there is a DNA polymerase
known as Taq polymerase which sustains in the heat and helps replicate
the DNA strands successfully. Before applying this, we add a primer to
each strand. This is the basic philosophy of PCR. This procedure is then
repeatedly carried out over each replicate until we get a huge number of
replicate DNA fragments. We get an exponential increase in copy number
of DNA.

In normal PCR, at a time only one sample of DNA sequence can be
amplified. In multiplex PCR, we can simulateneously amplify many targets
in one reaction by using multiple primer pairs. Cross primer template
binding can hamper this process, so we need to design the primers
efficiently. The sequences we take for amplification together should be
of least similarity.

Ancient DNA is often highly degraded, and even exceptionally preserved
permafrost specimens may contain only \(5\%\) of surviving DNA fragments
longer than 300 base pairs. Thus when PCR targets DNA sequences with
length \(> 300bp\),then it would mainly amplify the modern DNA. To
overcome this, a series of overlapping primer sets can be used to obtain
a long stretch of continuous DNA sequence in small, stepwise fragments.
It is also routine to clone at least some of the amplification products
of aDNA experiments, as this can identify potential contaminants or PCR
artifacts and allow evaluation of the extent of post-mortem damage.

For DNA cloning, we first cut the part of the DNA we want to study. This
cutting of DNA is performed by restriction enzymes. These DNA cuts are
attached to carriers called vectors (molecules that can replicate and
when they replicate recreate the fragment inserted). Vectors are
inserted in host cell/living cell in a bacterium (in vivo). Then the
vector will find right condition to replicate and thus creates
moreclones of the DNA fragment we want. To see differences between DNA
cloning and DNA amplification by PCR, check this
\href{http://www.majordifferences.com/2013/10/difference-between-gene-cloning-and-pcr.html\#.VqmE4ZMrL-Y}{site}.

However, each PCR requires at least one template molecule of the desired
genomic region. Depending on DNA preservation, more or less DNA extract
will be required to begin each amplification reaction. Multiple
amplifications of short, overlapping fragments may be necessary to
reconstruct long, informative DNA sequences. This process is performed
in replicate to account for possible sequence errors due to miscoding
lesions in the template molecule.

For many studies, even the smallest lane on any next generation
sequencing (NGS) instrument will produce excessive amounts of sequence
data for a single sample. If larger numbers of samples are to be
analyzed, the cost soon becomes prohibitive if a full single lane is
used per sample. It is therefore important to be able to pool multiple
samples and sequence them in a single lane. As the information about
sample origin of individual sequence reads is lost in all NGS
approaches, this requires barcoding techniques, in which a specific tag
is attached to all DNA fragments allowing them to be sorted
bioinformatically after sequencing.

The most efficient way to produce a barcoded sequencing library is to
amplify a genomic target region using polymerase chain reaction (PCR)
with target-specific primers that include a sequencing adapter and
barcode tail.

\end{document}
